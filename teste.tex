\documentclass[pdftex,12pt,a4paper]{report}
\usepackage[portuguese]{babel}
\usepackage[pdftex]{graphicx}
\usepackage{hyperref}
\begin{document}
\author{pedro Maia}


\maketitle
\tableofcontents


\section{Pedro maia}


A deflagração da Revolução \textbf{liberal de 1820 no Porto}, com a \textit{rápida adesão de Lisboa} e o resto do país, obrigou o pai de Pedro I a retornar a Portugal em abril de 1821, deixando-o para governar o Brasil como regente.

Teve de lidar com as ameaças de revolucionários e com a insubordinação de tropas portuguesas, as quais foram, no entanto, todas subjugadas. A tentativa do governo português de retirar a autonomia política que o Brasil gozava desde 1808 foi recebida com descontentamento geral. Pedro I escolheu o lado brasileiro e declarou a independência do Brasil de Portugal em 7 de setembro de 1822. Em 12 de outubro foi aclamado imperador brasileiro e, em março de 1824, já havia derrotado todos os exércitos leais a Portugal. Poucos meses depois, Pedro I esmagou a Confederação do Equador, principal reação contra a tendência absolutista e a \textbf{política centralizadora de seu governo}.

\textit{Os Miseráveis do mundo!!}


\end{document}
